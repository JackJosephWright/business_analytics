\documentclass[]{tufte-handout}

% ams
\usepackage{amssymb,amsmath}

\usepackage{ifxetex,ifluatex}
\usepackage{fixltx2e} % provides \textsubscript
\ifnum 0\ifxetex 1\fi\ifluatex 1\fi=0 % if pdftex
  \usepackage[T1]{fontenc}
  \usepackage[utf8]{inputenc}
\else % if luatex or xelatex
  \makeatletter
  \@ifpackageloaded{fontspec}{}{\usepackage{fontspec}}
  \makeatother
  \defaultfontfeatures{Ligatures=TeX,Scale=MatchLowercase}
  \makeatletter
  \@ifpackageloaded{soul}{
     \renewcommand\allcapsspacing[1]{{\addfontfeature{LetterSpace=15}#1}}
     \renewcommand\smallcapsspacing[1]{{\addfontfeature{LetterSpace=10}#1}}
   }{}
  \makeatother

\fi

% graphix
\usepackage{graphicx}
\setkeys{Gin}{width=\linewidth,totalheight=\textheight,keepaspectratio}

% booktabs
\usepackage{booktabs}

% url
\usepackage{url}

% hyperref
\usepackage{hyperref}

% units.
\usepackage{units}


\setcounter{secnumdepth}{-1}

% citations


% pandoc syntax highlighting

% longtable

% multiplecol
\usepackage{multicol}

% strikeout
\usepackage[normalem]{ulem}

% morefloats
\usepackage{morefloats}


% tightlist macro required by pandoc >= 1.14
\providecommand{\tightlist}{%
  \setlength{\itemsep}{0pt}\setlength{\parskip}{0pt}}

% title / author / date
\title{Google Gstore Revenue Prediction}
\date{}


\begin{document}

\maketitle




\hypertarget{abstract}{%
\subsection{Abstract}\label{abstract}}

``Half the money I spend on advertising is wasted, the trouble is I
don't know which half'' -John Wanamaker

One of the largest problems in running a business is identifying who is
likely to spend money at your business. Online business' have access to
far more data on who is visiting their `store' than a brick and mortar
business, who might only have access to a small amount of data on people
who entered the store, but did not buy anything. \emph{SOME BRIDGE
SENTENCE HERE} The dataset we are analyzing is a Google Merchandise
Store (also known as GStore, where Google swag is sold) customer
dataset. The raw dataset contains 12 columns to predict the transaction
revenue per customer. The outcome from this analysis will aid in better
use of marketing budgets. This paper will model the customer data, and
create predictive models to estimate revenue generated from a holdout
set.

\hypertarget{introduction-and-background}{%
\subsection{Introduction and
background}\label{introduction-and-background}}

Identifying customers who will purchase a product after visitng a
website is one of the most important factors in online commerce. The
Pareto Principle, a business principle that asserts 80\% of outcomes
result from 20\% of causes . In our case this means that we would expect
80\% of the revenue to be generated by 20\% of the customers {[}1{]}.
The data from the Gstore contains 55 variables for around 900,000 site
visits. A rigorous investigation of these determinants is needed to
identify if this rule of thumb holds, and which customers marketing
resources should be put towards.

The Pareto Principle was developed by Italian economist Vilfredo Pareto,
who discovered that 80\% of the land in Italy was owned by 20\% of the
population. This is a description of a power law distribution, known as
the Pareto distribution. It has been found that natural phenomena
exhibit this distribution, particulary in commerce. This was translated
to business administration by consultant Joseph M. Juran, who urges
decision makers to find ``their golden 20\%''.{[}1{]} Where they can
make actionable statements like ``These are my top transaction
locations'' or ``these are my top traffic channels for revenue
generation.''

A lay approach to identifying the golden 20\% is to graph a single
categorical variable and identify which category most of the revenue was
generated from. (chart 1) This however makes it more difficult to
address variable interaction.

The purpose of this paper is to generate a multivariate model of
transaction behavior of the Gstore visitors so that we can identify
customers most likely to generate revenue using data from the Google
Gstore. In particular, it develops a Random Forest model to predict the
revenue generated by a specific transaction. Random Forest Regression is
a supervised learning algorithm that uses ensemble learning method for
regression. Ensemble learning method is a technique that combines
predictions from multiple machine learning algorithms to make a more
accurate prediction than a single model. We will compare this with a
generalized linear model that generalizes linear regression by allowing
the linear model to be related to the response variable via a link
function (gaussian) and by allowing the magnitude of the variance of
each measurement to be a function of its predicted value.

Each model is evaluated using the Root Mean Squared Error (RMSE)
function. The model with the lowest RMSE is chosen.

This paper reports which factors we deem to be the most important,
providing insights for a better decision making model for how to target
specific customers to an online store.

The organization of this paper will be: -description of the
methodological framework -description of the Gstore data -empirical
results of the modeling -conculsions drawn from our findings

\hypertarget{methodology}{%
\subsection{Methodology}\label{methodology}}

Modeling transaction revenue can be either broken down as a two step
process (identifying discrete outcomes, then modeling revenue of the
positive class) , or creating a randomForest regression model

We expect Random Forest modeling to work better if: 1. The underlying
function is not truly linear 2. We end up selecting a high number of
predictor variables 3. There is high covariate shift (the distributions
shift between the train and the test set){[}6{]}

While the Random Forest Regression seems like the best fit for our data,
we will employ the `no free lunch' theorem {[}7{]} and generate linear
models as well for comparison.

We will be predicting the natural log of the sum of all transactions per
user (further discussed in the data section)
\[y_{user} = \sum_{i=1}^{n} transaction_{user_i}\]

\[target_{user} = \ln({y_{user}+1})\]

\hypertarget{data}{%
\subsection{Data}\label{data}}

Data analyized in this paper as collected from \emph{TIMEFRAME} for
assesment of the Gstore. The data provided is for all visitors to the
Google gstore during this period.

There are almost 1 million rows and 55 columns once unpacked, The
following is a description of the most consequential columns from the
data.

fullVisitorId: - A unique identifier for each user of the Google
Merchandise Store. This will be important for summing each transactors
behavior since it is constant between a users unique visits over time.
channelGrouping: - The channel via which the user came to the Store.
This is where the visitors to the Gstore ``came from'' on the internet.
Some examples of categoreis are `Social' (social media links), Referral
or Organic Search. date: - The date on which the user visited the Store.
device: - The specifications for the device used to access the Store,
such as tablet, phone or desktop. geoNetwork: - This section contains
information about the geography of the user, such as the country, state
and city from which they accessed the store. socialEngagementType:
-Engagement type, either ``Socially Engaged'' or ``Not Socially
Engaged'', referring to social media usage totals: -This section
contains aggregate values across the session. trafficSource: -This
section contains information about the Traffic Source from which the
session originated. visitId: - An identifier for this session. This is
part of the value usually stored as the \_utmb cookie. This is only
unique to the user. For a completely unique ID, you should use a
combination of fullVisitorId and visitId. visitNumber: - The session
number for this user. If this is the first session, then this is set to
1. visitStartTime: - The timestamp (expressed as POSIX time). hits: -
This row and nested fields are populated for any and all types of hits.
Provides a record of all page visits. customDimensions: - This section
contains any user-level or session-level custom dimensions that are set
for a session. This is a repeated field and has an entry for each
dimension that is set. totals - This set of columns mostly includes
high-level aggregate data

The most notable transformation that is employed is the natural log of
the target revenue. As shown in in chart 2, the transactions are on the
whole all zero, causing a severe right skew. Taking the log of revenue
makes the data more normally distributed, allowing for more precise
analysis. This also gives us our first insights into finding the golden
20\%, since most of the visitors to the Gstore are not revenue
generators.

Another notable feature of the data is that 40\% of the variables are
either sparse or only contain one value, which can be excluded from our
analysis for conatining no row-wise information.

\hypertarget{empirical-results}{%
\subsection{Empirical Results}\label{empirical-results}}

A randomForest model was used to optimize the RMSE and select the
optimal number of predictors and node size for our decision trees.
Random Forest is a bagging technique that contains a number of decision
trees on various subsets of the given dataset and takes the average to
improve the predictive accuracy of that dataset.'' Instead of relying on
one decision tree, the random forest takes the prediction from each tree
and based on the majority votes of predictions, and it predicts the
final output. The greater number of trees in the forest leads to higher
accuracy and prevents the problem of overfitting. {[}11{]} Using
parallel processing 10 models are trained across a grid of 20 to find
the best hyperparameters for the random forest model. The optimal Random
Forest regression obtained a .913 RMSE, while the GLM RMSE is 1.044

In chart 3, it is clear that the random forest model is more narrow in
its distribution of predicted values, leading to a lower RMSE. Table 1
is the correlation matrix of select variables included in the modeling.
A correlation matrix will give decision makers an intuitive
understanding of the different factors on purchasing behavior, as well
as indication of possible predictors for revenue generating transactors.
For example, device category (which hardware people use to access the
Gstore) is highly correlated with revenue, as the majority of revenue
has come from desktop users, while SOMETHING ELSE does not show high
correlation with the target variable, and is likely to be eliminated in
the model building process

Table 2 is the feature importance output from our Random Forest
regression modeling. In terms of decision tree modeling, the higher up
the branch network, the more impact the variable has to selecting which
leaf the data belongs to. The ten most important variables for
classification are, in order: -hits -pageviews -source -channelGrouping
-medium -metro -operating system -deviceCategory -region -isMobile

In Table 2, we show the differences in in estimation between a GLM and a
Random Forest Regression. In this table we report the estimation results
of both models. THE RMSE metric reports the differences between values
predicted by a model and the values observed. In our case this is total
revenue per customer. RMSE is a measure of accuracy that is useful in
comparing different models

RMSE can be calculated as follows:

\[ RMSE = \sqrt{\frac{1}{n}\Sigma_{i=1}^{n}{\Big(\frac{\hat{y}_i -y_i}{\sigma_i}\Big)^2}}\]

The predictor with the most impact was hits. This intuitively makes
sense as this is the most direct measure for interest in the Gstore that
we have in the dataset. This also goes for pageviews, and could be
combined into an engagement metric for decision makers. Because most
pageviews are low, the log of these predictors allowed us to generate
more detail in our model. \emph{calculate what percent of revenue came
from top 20\% pageviewers}

The next group of predictors are TrafficSource and channelGrouping. Both
of these variables tackle where the visitor came from on the internet.
The largest share of revenue came from referral. Out of the possible
places a visitor could have come from, referral conveys the most
interest in the product before the visit even starts. Recent studies
from Neilsen show that ``\ldots referrals are the most trusted form of
advertising, with 92\% of consumers reporting they completely or
somewhat trust referrals from people that they know.''{[}8{]}

The predictor metro is next up in terms of importance. In a recent CBS
news report, more than 90\% of tech jobs are located in just 5 US
cities. It would make sense that someone who wished to purchase Google
products would be in some way involved in technology. {[}9{]} This
predictor falls in the same importance band as operatingSystem,
deviceCategory, region and isMobile. As is to be expected, most of the
revenue comes from American Google Chrome users. Chrome is a product of
google, which is a country located in America.

Surprisingly, even though Macintosh operating systems are second to
Windows OS users, Mac OS users generate far more revenue than almost
every other category combined. This might be due to the high brand
identification that Apple has generated for its products. Marketer Marc
Globe, author of \textbf{Emotional Branding} had this to say about Apple
users. This high rate of revenue from Apple users could be indicative
that the brand loyalty is transferred to Google products.

``Apple's brand is the key to its survival. It's got nothing to do with
innovative products like the iMac or the iPod{[}10{]}.''

\hypertarget{summary-and-conclusions}{%
\subsection{Summary and Conclusions}\label{summary-and-conclusions}}

In this paper, we compare the results of a GLM and Random Forest
Regression model to accurately predict the revenue generated by a
visitor to the Google Gstore. Several important factors are found to
influence a Gstore visitor's decision to make a purchase. The most
important predictors point us towards visitors who are cosmopolitan
Americans emotionally invested in technology. They were also very likely
to be referred to the Gstore by someone they trust. Our predictors line
up with business intuition, particuarly the 80 20 rule. **GET SOME STAT
FROM THE DATA PROVING 80 20*

This study generates useful insights for a better understanding of
customer behavior on the Google Gstore. For decision makers, this could
lead to more targeted marketing, and even what data is important to
collect from a visitor to an online market.

\hypertarget{references}{%
\subsection{References}\label{references}}

{[}1{]} Carla Tardi, 2020. 80-20 Rule.
\url{https://www.investopedia.com/terms/1/80-20-rule.asp}

{[}6{]}
\url{https://www.seldon.io/what-is-covariate-shift/\#}:\textasciitilde:text=Covariate\%20shift\%20is\%20a\%20specific,training\%20environment\%20and\%20live\%20environment.\&text=Covariate\%20shift\%20is\%20also\%20known,issue\%20encountered\%20in\%20machine\%20learning.

{[}7{]}
\url{https://machinelearningmastery.com/no-free-lunch-theorem-for-machine-learning/}

{[}8{]}
\url{https://www.springboard.com/blog/business-and-marketing/referral-marketing-what-it-is-and-why-it-works/\#}:\textasciitilde:text=A\%20recent\%20Nielsen\%20study\%20shows,referrals\%20from\%20people\%20they\%20know.\&text=Referral\%20marketing\%20leads\%20have\%20been,generated\%20from\%20other\%20marketing\%20channels.

{[}9{]}
\url{https://www.cbsnews.com/news/90-percent-of-tech-jobs-growth-concentrated-in-just-5-cities-according-to-brookings-institute-report/}

{[}10{]}
\url{https://www.wired.com/2002/12/apple-its-all-about-the-brand/}

{[}11{]}
\url{https://medium.com/geekculture/xgboost-versus-random-forest-898e42870f30}



\end{document}
